\documentclass[12pt]{article}

\textwidth=6in
\textheight=9in
\parindent=0in
\parskip=1ex
\voffset=-1in
 
\input{epsf}
 
\newcommand{\bi}{\begin{itemize}}
\newcommand{\ei}{\end{itemize}}
\newcommand{\bd}{\begin{description}}
\newcommand{\ed}{\end{description}}
\newcommand{\be}{\begin{enumerate}}
\newcommand{\ee}{\end{enumerate}}

\newcommand{\btab}[1]{\begin{table}[htbp]\begin{center}\begin{tabular}{#1}}
\newcommand{\etab}[2]{\end{tabular}\caption{\label{#1} #2}%
                      \end{center}\end{table}}
\newcommand{\myfig}[4]{%
  \begin{figure}[htbp] \begin{center}%
  \makebox{\epsfysize=#2 \epsfbox{#1}} \caption{\label{#3} #4}%
  \end{center} \end{figure}}

\title{GAL implementation of the BDM interface}
\author{Thomas Walle}


\begin{document}

\maketitle

The BDM ({\it Background Debug Mode}) interface of the MC68332 is a
very powerful method to allow for debugging on the fly as well as an
efficient interface to download code.

Basically, it is a serial line that is driven from an external
clock. To save pin count on the chip, all but one signals have
different meanings in normal CPU mode. To avoid accidental entering of
the BDM mode, the BDM interface has to be enabled during
reset. Afterwards, the CPU enters BDM mode in the following 3 cases:

\bi
\item execution of Breakpoint operation
\item double bus fault
\item external signal {\sc /bkpt} active
\ei

The CPU signals entering of the BDM mode with {\sc freeze} going high.
The serial interface will be activated. In this case, {\sc /bkpt}
signal gets the function of the clock {\sc dsck}, {\sc /ifetch} gets
serial data input {\sc dsi} and {\sc /ipipe} gets serial data output
{\sc dso}.

The BDM interface is controlled via the parallel port of a PC. Its
control signals allow a fast and simple method to control the
interface.

\myfig{gal.eps}{70mm}{gal.fig}{Structure of the interface}

The interface, as it is shown in fig. \ref{gal.fig}, implements the
public domain BDM interface. It supports 3 basic functions:

\be
\item Performing a reset and enabling the BDM interface. Depending on
  the {\sc dsck} signal the CPU stops immediately before the first fetch or it
  runs through and boots as usual.
\item Activation of the BDM mode. Thus, commands can be sent to read
  or set memory locations, registers, CPU state, ...
\item Single stepping the CPU. After sending a GO command in BDM mode
  the CPU will resume normal execution. The corresponding pins get their
  normal function back. {\sc /ifetch} ({\sc dsi}) indicates the first
  fetch after resuming normal mode. Hence, if {\sc step} is activated,
  this causes once again activation of {\sc /bkpt}.  
\ee

The following equations describe the implementation of the interface
via a GAL.

\small
\begin{verbatim}
MODULE bdm

/* Pinout */
            -------
       CK  | 1   24|  VCC
      RES  | 2   23|  CKO
    /DSCK  | 3   22|  FR-OUT
  /DSI-IN  | 4   21|  DSO-OUT
    /STEP  | 5   20|  LED0
           | 6   19|  LED1
           | 7   18|  LED2
    LEDEN  | 8   17|  LED3
           | 9   16|
    FR-IN  |10   15|  /BKPT
   DSO-IN  |11   14|  DSI-OUT
      GND  |12   13|
            -------




/* Equations */

BKPT = RES + /DSCK + Q;

Q := 1;
Q.AR = /STEP;
CKO = /DSI-OUT + FR-IN;

DSI-OUT = DSI-IN;
DSI-OUT.OE = FR-IN;
DSO-OUT = /(DSO-IN * FR-IN);

FR-OUT = FR-IN;

LED0 = FR-IN;
LED1 = BKPT;
LED2 = RES;
LED3 = STEP;
LED0.OE = LEDEN;
LED1.OE = LEDEN;
LED2.OE = LEDEN;
LED3.OE = LEDEN;
\end{verbatim}
\normalsize

The first function is simply implemented by connecting {\sc /res} to the
equation for {\sc /bkpt}.

To stop the normal execution of the CPU and enter BDM mode the
{\sc dsck} line is held low until the CPU stops which is indicated by
{\sc fr-in} going high. In this case. {\sc /ifetch} output becomes now
the serial data input {\sc dsi}, thus the output driver of {\sc dsi-out}
is enabled. The communication is fully bidirectional. Data lines are
changed with a low to high transition of {\sc dsck} and have to be
latched with the high to low transition. At the same time a new
command is sent via the {\sc dsi} line, the possible values of a
previous command are sent via {\sc dso}. The CPU returns from BDM mode
after transmission of a GO command.

For the single stepping function it is important to notice that the GO
command ends with a 0, i.e. the last bit sent causes {\sc dsi} line to be
low. After resuming normal mode (indicated by {\sc fr-in} going low) the
{\sc /ifetch} function is turned on again. Hence, for a short while this
line is floating before it is driven high (inactive) again. With a
high to low transition of {\sc /ifetch} the CPU indicates a fetch bus
cycle. In order to stop the CPU after this command again, this is the
minimum period to wait for a new activation of {\sc /bkpt}. To avoid
unintentional activation of {\sc /bkpt} during the floating period of
{\sc /ifetch} the line is held low by a pull down resistor. To catch the
negative edge of {\sc /ifetch} we use the negated signal as clock for
register Q. If clocking is enabled through releasing of reset,
i.e. {\sc step} going high, a 1 is clocked into Q causing activation of
{\sc /bkpt} again. In order to have no interference with other functions as
well as the normal execution mode of the CPU the {\sc step} signal has to be
low in every other case.

The physical implementation of the interface is very simple by the use
of a GAL. Nevertheless, there is a problem which is caused by a
usually too weak output driver of the parallel port. To have a
convenient BDM interface the cable to the CPU can reach up to a length
of 2m. The fast response in case of the single stepping functionality
makes it desirable to have the GAL close to the CPU. Thus, a weak port
output driver has a tremendous rise time in its signals which causes
oscillating output signals of the GAL and hence destroys the whole
transmission. To overcome this, a Schmitt trigger inverter was
inserted for all signals going from the line printer output port to
the GAL (cf. \ref{gal.fig}). Since this inverter is available in
surface mount technology, it can be hidden in the plug for the
parallel port.

\btab{|c||c|c|c|c|}
\hline
    & \multicolumn{4}{c|}{Port bytes} \\ \cline{2-5}
Bit & +0 & +1 & +2 & +3 \\ \hline\hline
0x01 &      & -      & /strobe &      \\ \cline{3-4}
0x02 &      & -      & /afeed  &      \\ \cline{3-4}
0x04 &      & -      & /init   &      \\ \cline{3-4}
0x08 & not  & /error & /selin  & not  \\ \cline{3-4}
0x10 & used & -      & -       & used \\ \cline{3-4}
0x20 &      & pend   & -       &      \\ \cline{3-4}
0x40 &      & /ack   & -       &      \\ \cline{3-4}
0x80 &      & /busy  & -       &      \\ \hline
\etab{lptport}{Signals of the line printer port (LPT)}

Finally, table \ref{lptport} gives an overview of the logical
interface of the parallel port indicating the bit positions of all
used signals.

\end{document}
