\documentclass[12pt]{article}

\textwidth=6in
\textheight=9in
\parindent=0in
\parskip=1ex
\voffset=-1in
 
\input{epsf}
 
\newcommand{\bi}{\begin{itemize}}
\newcommand{\ei}{\end{itemize}}
\newcommand{\bd}{\begin{description}}
\newcommand{\ed}{\end{description}}
\newcommand{\be}{\begin{enumerate}}
\newcommand{\ee}{\end{enumerate}}

\newcommand{\isc}{$I^2C$}
\newcommand{\btab}[1]{\begin{table}\begin{center}\begin{tabular}{#1}}
\newcommand{\etab}[2]{\end{tabular}\caption{\label{#1} #2}%
                      \end{center}\end{table}}
\newcommand{\myfig}[4]{%
  \begin{figure}[htbp] \begin{center}%
  \makebox{\epsfysize=#2 \epsfbox{#1}} \caption{\label{#3} #4}%
  \end{center} \end{figure}}
\newcommand{\beqn}{\begin{eqnarray*}}
\newcommand{\eeqn}{\end{eqnarray*}}

\title{Interface to \isc\ bus driver and DUART}
\author{Thomas Walle}

\begin{document}

\maketitle

The \isc\ bus driver and the DUART are mentioned to work with the MC68000 bus.
Since the MC68332 has the same interface but is much faster, problems in the
timing arise. In order to overcome these problems we embed the two
components, i.e. we place a bidirectional driver into the data bus, which is
controlled by a programmable logic device (GAL22V10). This device handles
also the proper execution of the interupt acknowledge cycle.

Fig. \ref{bus-embed} shows the construction of the embedding. In the lower
right part are the two \isc bus controllers (PCF8584) and the DUART 
(SCN68681). The GAL reads the bus control signals and controls the data bus
driver (ABT652). In addition, it interfaces the data acknowledge signal
to the CPU (/DACKO). In the upper part of the figure is a block for fast
binary output depicted. With help of an address comparator (ALS518) the GAL
decodes the write access to the register (ABT16823) and generates its clock.
Eight of the outputs serve as direction control for two dual motor controls.
The other eight outputs are unbound and can serve as fast binary outputs.

\myfig{bus-embed.eps}{150mm}{bus-embed}{Overview of the embedding}

In the first section, we describe the control equations of the GAL for both
blocks. In section two, we deal with the embedding of the two kinds of 
busmaster chips. Then, in the third section we show the details of the 
implementation of the binary output block.

\section{General control}

The problem of the externel devices consists in the possibly very long
disable time of the data acknowledge signal (DSACK) and the data bus for
read accesses. Furthermore, the DUART has a very long hold time for the
data in a write access.

To overcome this, we insert a bidirectional driver with internal registers
into the data bus. Each port of the driver is controlled by 3 signals:

\bd
\item[SRCSEL] selects from source the output is fed --- the internal
  register for this direction or the other port
\item[CLK] clocks data from the other port into the internal register
\item[/OE] output enable for the port
\ed

The CPU side port is activated for read accesses by /RDOE. Basically, 
it runs with the according chip select.

A direct write after an read access to some external bus master would
due to the large disable time conflict with the still driving master
and cause bus contention. But luckily there has to be a fetch after
the access and in the meantime the external bus has disabled its data
bus interface. Even for fast loops, where there is no fetching, the
next cycle is completely internal (update and test internal counter).

In write cycles the chip select is enabled as early as the data strobe
(/DS). Since the computation of the /WROE takes some time and the
external bus controller expects valid data with the chip select going
low, we enable the data bus driver with the write signal (/RW), that
is valid after the first half clock cycle of the access. Consequently,
in every write cycle the data bus driver is enabled. To ensure the
large hold time specification of the DUART the /WROE signal is
registered. Since the RW signal changes its state as early as in the
first clock cycle of the next bus cycle, the data port is still active
one clock cycle after the write access. To deliver valid data as long
as possible, the port's source is switched to the internal register
with the negated /AS signal. Thus, data is valid early in the
beginning of the write cycle and gets invalid when the next read cycle
starts.

The data acknowledge signal is handled in the same way as the data bus is.
Since it is unidirectional, the driver is implemented in the GAL.
The computattion of the enable signal needs to cover several cases and
thus, the logic equation contains some 'or'. But the GAL offers only
one term. Consequently, a seperate output is used for its computation.

A second issue is the interupt acknowledge cycle. It is basically a read
cycle, but there is no chip select set. Hence, the cycle has to be decoded
by the GAL. Such a cycle is recognized by the two low order function code
bits (FC1, FC0) beeing 1 and address bit A0 beeing 1 as well. Address bits
A1,\ldots ,A3 give the interupt level. This is sufficient because the lowest
address bit distinguishes between other CPU space cycles. Due to lack
of input pins A0 is not decoded. In selecting only unequal interupt
levels, i.e. A1 is always 1 in this case, this distinguishes the
interupt acknowledge cycles for these devices from other CPU cycles.

Finally, the resulting equations for the GAL.

\small
\begin{verbatim}
GAL22V10

CK EQ RW /AS FC1 FC0 A3 A2 A1 /CS5 /CS2 GND
/CS1 /DACKI DACKOE /IACK1 /IACK3 /IACK5 RDOE /WROE WRCK /REGCE /DACKO VCC


RDOE = (/CS1 + /CS2 + /CS5)*RW + /IACK1 + /IACK3 + IACK5;

/IACK1 = /AS*RW*FC1*FC0*/A3*/A2*A1;    //IRQ1 --> I2CB
/IACK3 = /AS*RW*FC1*FC0*/A3*A2*A1;     //IRQ3 --> I2CA
/IACK5 = /AS*RW*FC1*FC0*A3*/A2*A1;     //IRQ5 --> DUART

/WROE := /RW;

WRCK = (CS1 + CS2 + CS5)*/RW;

DACKOE = /CS1 + /CS2 + /CS5 + /IACK1 + /IACK3 + /IACK5;
/DACKO.OE = DACKOE;
/DACKO = /DACKI;

//----------------------------------------------------------

/REGCE := /AS*/RW*/FC1*FC0*EQ*REGCE    //on with adr valid and /AS
       +     /RW*/FC1*FC0*EQ*/REGCE;   //off with read or adr invalid
\end{verbatim}
\normalsize

In the lower part of the definition of the GAL-equations the clock enable
for the binary output is given. The register is clocked with the /DS signal.
The clock is enabled after recognition of a valid address and an active
address strobe (/AS). It is disabled again when the address gets invalid or
the next read cycle starts. This is certainly before the next clock.

The GAL registers are automatically reset after during power up. Thus, the
write enable of the data bus driver and the clock enable of the binary
output register are both invalid.


\section{Timing of the accesses to the external busmasters}

The accesses to external devices are mainly determined by two signals
-- the chip select (/CS) and the data acknowledge (/DACK). An access
runs basically in 3 steps:

\be
\item The CPU starts the access by preparing all necessary signals on
  the bus and then pulling /CS low.
\item The external device stores the data from the data bus or puts
  valid data on the data bus and indicates that itself is ready now in
  pulling /DACK low.
\item The CPU finishes the access in disableing /CS again. It expects
  the device to disable its /DACK and its data bus interface soon
  thereafter.
\ee

Out of this observation, the analysis of the timing will be divided
into 3 time points, too. At the first point, there are limitations due
to the setup times to the activating edge of /CS, i.e. for the
addresses, the RW signal and in case of a write also for the data.

From the activating of /CS results a propagations delay for the
activating edge of /DACK. This gives the basis for the calculation of
the number of wait states. At the same time, there are setup and hold
time limitations to a certain edge of the internal clock of the CPU in
the case of read accesses.

At the third point, the disableing of /CS, the device has to obey a
disable time of /DACK and the data bus in case of a read. In case of a
write, there are hold time limitations of the device for addresses,
data and the RW signal.

Before we verify all the limitating equations in detail, we first
compute some basic time points, that a common to all cases.

\begin{table}
\begin{center}
\scriptsize
\begin{tabular}{|l|l||c|c||c|c||c|c|}
\hline
Symbol & \multicolumn{1}{c||}{Characteristic} & \multicolumn{2}{c||}{16.78MHz} &
\multicolumn{2}{c||}{20.93MHz} & \multicolumn{2}{c|}{25.17MHz} \\
&& min & max & min & max & min & max \\ \hline\hline
$t_{chav}$ & clock high to address, FC valid & 0 & 29 & 0 & 23 & 0 &
19 \\ \hline
$t_{clsa}$ & clock low to /as, /ds, /cs asserted & 2 & 25 & 0 & 23 & 0
& 19 \\\hline
$t_{avsa}$ & address, FC valid to /as, /cs asserted & 15 & - & 10 & -&
8 & - \\\hline
$t_{clsn}$ & clock low to /as, /cs, /ds negated & 2 & 29 & 2 & 23 & 2
& 19 \\\hline
$t_{snai}$ & /as, /ds, /cs negated to address, FC invalid & 15 & - & 
10 & - & 8 & - \\ \hline
$t_{snrn}$ & /as, /ds, /cs negated to rw negated & 15 & - & 10 & -
& 10 & - \\ \hline
$t_{chrl}$ & clock high to rw low & 0 & 29 & 0 & 23 & 0 & 19 \\ \hline
$t_{raaa}$ & rw asserted to /as, /cs asserted & 15 & - & 10 & - & 10 &
- \\
\hline
$t_{rasa}$ & rw low to /ds, /cs asserted (write) & 70 & - & 54 & -
& 40 & - \\ \hline
$t_{chdo}$ & clock high to data out valid & - & 29 & - & 23 & - & 19 \\
\hline
$t_{sndoi}$ & /ds, /cs negated to data out invalid & 15 & - & 10 & -
& 5 & - \\ \hline
$t_{dvsa}$ & data out valid to /ds, /cs asserted (write) & 15 & - & 10
& - & 8 & - \\ \hline
$t_{dicl}$ & data in valid to clock low & 5 & - & 5 & - & 5 & - \\ \hline
$t_{sndn}$ & /as, /ds negated to /dsack negated & 0 & 80 & 0 & 60 & 0
& 50 \\
\hline
$t_{sndi}$ & /ds, /cs negated to data in invalid & 0 & - & 0 & - & 0 &
- \\
\hline
$t_{shdi}$ & /ds, /cs negated to data in High-Z & - & 55 & - & 48 & -
& 45 \\
\hline
$t_{aist}$ & asynchronous input setup (/dsack) & 5 & - & 5 & - & 5 & -
\\
\hline
$t_{radc}$ & rw asserted to data bus impedance change & 40 & - & 32 &
- & 25 & - \\ \hline
\end{tabular}
\caption{\label{proc-tim} Bus timing specification of the MC68332}
\end{center}
\end{table}

Table \ref{proc-tim} gives a summary of all used timing information of
the CPU. Since there are 3 types of CPU possible in the system, there
are 3 columns with the respective figures.

For the symbols in the equations, we introduce the following
notation. The minimum propagation delay is always denoted with a minus
sign, whereas the maximum with a plus sign. Where necessary the pair
of input and output port meant is given in round brackets and the
subscript 'pd' is used. Setup times (with the subscript 's') are
always maxmimum times and hold times (with the subscript 'h') are
always minimum times. Where necessary the input signal meant is given
in round brackets. Disable times are handled like propagation
delays. To avoid naming conflicts, the device name to which the time
refers is given in square brackets.

In starting an access to an external device the CPU prepares the bus,
i.e. it drives address bus and RW signal in the first half of the
first CPU cycle (S0).

\beqn
t_{adr} &=& t_{chav}\\
t_{rw} &=& t_{chrl}\\
\eeqn

In the second half of the first CPU cycle (S1), valid addresses and a
valid RW signal are indicated by activation of /AS.

\[ t_{as} = T/2 + t_{clsa} \]

For a read access the chip select follows the /AS signal ($t_{cs} :=
t_{as}$), whereas it follows the /DS signal for write accesses ($t_{cs}
:= t_{ds}$). The indication of valid data through activation of /DS
happens in the second half of the second CPU cycle (S3).

\[ t_{ds} = 3/2 T + t_{clsa} \]

In this case the data bus is valid at latest at

\[ t_{dat}^+ = t_{cs}^+ - t_{vdsa}^- = 3/2 T + t_{clsa}^+ - t_{vdsa}^- \]

In order to keep the following investigation clear, we divide it
further into the two cases read access and write access.

\btab{|l|l|c|c|}
\hline
Symbol & \multicolumn{1}{c|}{Characteristic} & min & max \\
\hline\hline
\multicolumn{4}{c}{GAL22V10DQP} \\ \hline\hline
$t_{pd}(i,o)$ & input valid to output valid & 1 & 10 \\ \hline
$t_{pd}(ck,o)$ & clock to output valid & 1 & 7 \\ \hline
$t_s$ & data setup to clock & - & 7 \\ \hline
$t_h$ & data hold to clock & & - \\ \hline
$t_{en}$ & input to output enabled & 1 & 10 \\ \hline
$t_{dis}$ & input to output disabled & 1 & 9 \\ \hline\hline
\multicolumn{4}{c}{ABT652} \\ \hline\hline
$t_{pd}(i,o)$ & input valid to output valid & 1.5 & 6.7 \\ \hline
$t_{pd}(sel,o)$ & select valid to output valid & 1.5 & 7.7 \\ \hline
$t_{pd}(ck,o)$ & clock to output valid & 1.7 & 8.4 \\ \hline
$t_s$ & data setup to clock & - &  \\ \hline
$t_h$ & data hold to clock & & - \\ \hline
$t_{en}$ & /oe to output enabled & 1.3 & 8.5 \\ \hline
$t_{dis}$ & /oe to output disabled & 1.5 & 8.2 \\ \hline\hline
\multicolumn{4}{c}{ALS518} \\ \hline\hline
$t_{pd}(i,o)$ & input valid to output valid & 3 & 33 \\ \hline\hline
\multicolumn{4}{c}{ABT16823} \\ \hline\hline
$t_s(ck)$ & data setup to clock & - &  \\ \hline
$t_h(ck)$ & data hold to clock & & - \\ \hline
$t_s(en)$ & setup en to clock & - &  \\ \hline
$t_h(en)$ & hold en to clock & & - \\ \hline
\etab{logic-tim}{Timing specifications of the embedding logic}

Table \ref{logic-tim} gives all necessary specifications of the
embedding logic. Table \ref{dev-tim} summarizes all the timing
limations that have to be verified for the single busmaster devices.

\btab{|l|l||c|c||c|c||c|c|}
\hline
Symbol & \multicolumn{1}{c||}{Characteristic} & 
\multicolumn{2}{c||}{PCD8584} &
\multicolumn{2}{c||}{SCN68681} & \multicolumn{2}{c|}{XR68C192} \\
&& min & max & min & max & min & max \\ \hline\hline
$t_{avcl}$ & address valid to /cs low & 10 & - & 10 & - &&\\ \hline
$t_{wlcl}$ & rw valid to /cs low & 10 & - & 0 & - &&\\ \hline
$t_{dvcl}$ & data valid to /cs low & 0 & - & -261 & - &&\\ \hline
$t_{chai}$ & /cs high to address invalid & 0 & - &&&&\\ \hline
$t_{chwh}$ & /cs high to rw high & 0 & - & 0 & - &&\\ \hline
$t_{chdi}$ & /cs high to data invalid &&&&&& \\ \hline
$t_{cldl}$ & /cs low to /dtack low & - & 325 & - & 666 &&\\ \hline
$t_{chdh}$ & /cs high to /dtack high &  & 120 &  & 100 &&\\ \hline
$t_{cldv}$ & /cs low to data valid & & 180 & & 175 &&\\ \hline
\etab{dev-tim}{Timing specifications of the busmaster devices}


\subsection{Read accesses}

Figure \ref{rd-dev-tim} illustrates the read access graphically.

At the activating edge of /CS addresses and the RW signal have to be
stable a certain time before.

\beqn
10 = t_{avcl} &<& t_{avsa} = (15,10,8)\\
(10,0) = t_{wlcl} &<& t_{raaa} = (15,10,10)\\
\eeqn

A method to guarantee the setup time for the PCD8584 in th case of a
25MHz CPU
would delay the /CS artificially. Since this is a setup time and a
worst case analysis, we can also state the following. In normal cases,
i.e. proper power supply, decent ambient temperature, the worst case
will never happen. If it happens the CPU speed will be reduced
slightly.

For the recognition and proper save of the data value, there are two
times relevant: the time when the data bus is stable and the time when
/DACK is stable on the external bus.

The data bus driver ABT652 and the /DACKO output of the GAL are
enabled with the /CS. Since it takes quite a while until the busmaster
device generates a response, only the propagation delays of the
interface logic is of interest. In a bus cycle without wait states,
/DACK will be clocked at the beginning of S3 whereas the data bus will
be latched at the beginning of S5. With $i$ the number of wait state
cycles, the following has to be obeyed. For the data bus it is

\beqn
& & t_{dat}^+ + t_{dicl} < (5/2 + i) T\\
&\Leftrightarrow& t_{cs}^+ + t_{cldv} + t_{pd}^+(i,o)[GAL] +
t_{dicl} < (5/2 + i) T\\
&\Leftrightarrow& T/2 + t_{clsa^+} + t_{cldv} + t_{pd}^+(i,o)[GAL] +
t_{dicl} < (5/2 + i) T\\
&\Leftrightarrow& (i+2) T > t_{clsa}^+ + t_{cldv} + 6.7 + t_{dicl}\\
&\Leftrightarrow& (i+2)(60,50,40) - (25,23,19) > (191.7,188.7)\\
&\Rightarrow& i = (2,3,4)\\
\eeqn

and for the /DACK signal it is

\beqn
& & t_{cs}^+ + t_{cldl} + t_{pd}^+(i,o)[GAL] +
t_{aist} < (3/2 + i) T\\
&\Leftrightarrow& T/2 + t_{clsa}^+ + t_{cldl} + t_{pd}^+(i,o)[GAL] +
t_{aist} < (3/2 + i) T\\
&\Leftrightarrow& (i+1) T > t_{clsa}^+ + t_{cldl} +  6.7 + t_{aist}\\
&\Leftrightarrow& (i+1)(60,50,40) - (25,23,19) > (336.7,677.7)\\
&\Rightarrow& i = ((6,7,8)(11,14,17))\\
\eeqn

\myfig{rd-dev-tim.eps}{100mm}{rd-dev-tim}{Timing of a read access}

At the disableing edge of /CS there are 3 limitations. First, the data
bus has to be stable until after the hold time of the CPU
port. Second, the data bus and the /DACK signal must not be actively
driven until after the disable time of the CPU (which is roughly the
start of the next cycle).

The data bus is hold until the data bus interface chip is disabled
through the GAL.

\[ 0 = t_{sndi} < t_{pd}^-(i,o)[GAL] + t_{dis}^-[abt652] = 1.0 + 1.3 = 2.3
\]

At latest the data bus is disabled after the maxmimum times in the
equation above.

\[ (55,48,45) t_{shdi} > t_{pd}^+(i,o)[GAL] + t_{dis}^+[abt652] = 
10.0 + 8.5 = 18.5 \]

With the same edge also the /DACK is disabled. The disable time of the
output port of the GAL is enlarged by a small $\epsilon$ due to the
pullup resistor.

\[ (80,60,50) = t_{sndn}^+ > t_{dis}^+[GAL] + \epsilon = 9 + \epsilon \]


\subsection{Write accesses}

Figure \ref{rd-dev-tim} illustrates the read access graphically.

Since the RW signal is registered in the GAL to function as WROE, it
has to fulfill setup and hold times to the end of S1.

\[ t_{rw}^+ + t_s[GAL] < T \]
\[ \Rightarrow (60,50,40) = T > t_{chrl}^+ + 7.0 = (36,30,26) \]

As the RW signal is valid during the whole cycle, the hold time is
obviously obeyed.

We determine now, when the inner data bus DB is at latest stable. The
WROE is valid after the first CPU clock.

\[ t_{wroe}^+ = T + t_pd(ck,o)[GAL] = T + 7.0 \]

At the data bus interface chip there are now 3 concurrent paths,
i.e. from the WROE enableing the output, from the WSEL switching to
input port (instead of internal register) and the propagation from
input to output port.

\beqn
t_{db}^+ = max \{ && t_{as}^+ + t_{pd}^+(s,o)[abt652],  \\
&& t_{wroe}^+ + t_{pd}^+(oe,o)[abt652],\\
&& t_{dat}^+ + t_{pd}^+(i,o)[abt652] \} \\
\eeqn

\myfig{wr-dev-tim.eps}{100mm}{wr-dev-tim}{Timing of a write access}

With $T \ge 40$ the it follows that the last term determines the
maximum. Thus,

\[ t_{db}^+ = 3/2 T + t_{clsa} - t_{dvsa} + 6.7 \]

Now, we are ready to investigate the setup times at the activating
edge of /CS. For the addresses and the RW signal we get

\beqn
10 = t_{avcl} &<& t_{cs}^- - t_{adr}^+ = (63,52,41)\\
(10,0)= t_{wlcl} &<& t_{rasa} = (70,54,40)\\
\eeqn

Since the time, when the inner data bus DB is at latest stable is
determined by the propagation delay of the ABT652, this reduces the
minimum time span between data bus stable and activation of the /CS.

\[ (0,-261) = t_{dvcl} < t_{dvsa} - t_{pd}^+(i,o)[abt652] = 
t_{dvsa} - 6.7 = (8.3,3.3,1.3) \]

At the same time, the value of the data bus is stored in the internal
register of the ABT652. Therefore,

\[ t_s[abt652] < t_{dvsa} + t_{pd}^-(i,o)[GAL] \Leftrightarrow
(15,10,8) = t_{dvsa} > ... - 1.0 \]

The hold time is again obviously fulfilled.

For the recognition of the activation of /DACK the same equation hold
as in the case of a read access.

At the end of the write access, addresses, the RW signal and data have
to be held stable until after the hold time of the busmaster devices,
i.e.

\beqn
(0,?) = t_{chai} &<& t_{snai} = (15,10,8)\\
0 = t_{chwh} &<& t_{snrn} = (15,10,10)\\
? = t_{chdi} &<& t_h(db)\\
\eeqn

With the disactivating edge of /CS the data bus interface is switched
to its internal register. This allows to hold the internal data bus
until the next CPU starts

\beqn
t_h(db) &=& min \{ t_{as}^- , t_{wroe}^- \}\\
&=& \{ T/2 + t_{clsa}^- , T + t_{pd}^-(ck,o)[GAL] \}\\
&=& T/2 + t_{clsa}^- = (32,25,20)\\
\eeqn

For the disableing of /DACK the same equation hold as in the case of a
read access.


\subsection{Interupt request cycles}

An interupt request cycle is basically a read access to the device
with the only difference, that the CPU uses no /CS and therefore the
chip selects have to be decoded out of the address
bus. Instead of a chip select the device looks for a signal with
the same meaning on its /IACK port. Figure \ref{ir-dev-tim}
illustrates the access.

\myfig{ir-dev-tim.eps}{100mm}{ir-dev-tim}%
      {Timing of a interupt acknowledge cycle}

The differences in respect to the timing analysis only come only
through a little later start of the access out of the view from the
busmaster device. Because the /IACK has to
be decoded out of the address bits, the activating edge of /IACK is a
little later than an ordinary /CS would be. But this makes the setup
conditions on that edge even more relexed and the hold time conditions
have never been a problem. More precisely, we get

\beqn
10 = t_{avcl} &<& t_{avsa} + t_{pd}^-(i,o)[GAL] = (16,11,9)\\
(10,0) = t_{wlcl} &<& t_{raaa} + t_{pd}^-(i,o)[GAL] = (16,11,11)\\
\eeqn

Consequently, the device will generate a later response to that
starting edge, i.e.

\beqn
& & t_{cs}^+ + t_{cldl} + 2 \cdot t_{pd}^+(i,o)[GAL] +
t_{aist} < (3/2 + i) T\\
&\Leftrightarrow& T/2 + t_{clsa}^+ + t_{cldl} + 2 \cdot t_{pd}^+(i,o)[GAL] +
t_{aist} < (3/2 + i) T\\
&\Leftrightarrow& (i+1) T > t_{clsa}^+ + t_{cldl} +  13.4 + t_{aist}\\
&\Leftrightarrow& (i+1)(60,50,40) - (25,23,19) > (343.4,684.4)\\
&\Rightarrow& i = ((6,7,8)(11,14,17))\\
\eeqn

The rest of the access is completely congruent to the read access.


\section{Timing of the accesses to the binary output register}

The access to the fast binary output register ABT16823 is a simple
write access to a memory mapped device, i.e. the clock enable for the
register is decoded directly from the address bus. The register is
clock with the disableing edge of /DS. So, the setup time is
determined by the difference of the earliest disableing of /DS and the
latest time, when data is valid.

\[ t_s[abt16823] < T - t_{chdo}^+ + T/2 + t_{clsn}^- \]

The hold time is guarantied by the minimum data hold time of the CPU

\[ t_h[abt16823] < t_{sndoi}^- \]

The more critical observation is, whether the clock enable signal is
valid in time and is disabled again before the next CPU cycle starts.

In order to determine the timing of the clock enable, we distinguish
two cases: the earliest CPU cycle $i^-$ and the latest CPU cycle $i^+$
where the GAL might produce a valid clock enable. Thus,

\beqn
max \{ t_{as}^- , t_{rw}^- , t_{eq}^- , t_{adr}^- \} +
t_s[GAL] &\le& i^- T \\
max \{ t_{as}^+ , t_{rw}^+ , t_{eq}^+ , t_{adr}^+ \} +
t_s[GAL] &\le& i^+ T \\
\eeqn

with $t_{eq}^- = t_{adr}^- + t_{pd}^-[als518]$ and $t_{eq}^+ =
t_{adr}^+ + t_{pd}^+[als518]$. Since the minimum propagation delay of
the ALS518 is very short the earliest cycle is clearly determined by
the address strobe. Because $T/2 \ge 20$ and $t_{as}^-$ certainly less
than $T/2 - 7.0$ it follows that

\[ i^- = 1 \]

A similar observation yields to the result

\[ i^+ = 2 \]

With these precalculations the timing of the clock enable is now fixed
to

\[ t_{regce}^- = T + 1.0 \quad ; \quad t_{regce}^+ = 2T + 7.0 \]

In order to guarantee proper operation the latest clock enable has to
be earlier than the earliest clock, which is /DS.

\[ t_{regce}^+ < t_{ds}^- \Leftrightarrow 2T + 7 < 5/2 T + t_{clsn}^-
\Leftrightarrow T/2 > 7 - t_{clsn}^- \]

The same observation as for the enableing holds also for the
disabling, i.e. the disableing is certainly before the next active
edge of /DS.

\end{document}
